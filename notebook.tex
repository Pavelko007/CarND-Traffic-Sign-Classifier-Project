
% Default to the notebook output style

    


% Inherit from the specified cell style.




    
\documentclass[11pt]{article}

    
    
    \usepackage[T1]{fontenc}
    % Nicer default font (+ math font) than Computer Modern for most use cases
    \usepackage{mathpazo}

    % Basic figure setup, for now with no caption control since it's done
    % automatically by Pandoc (which extracts ![](path) syntax from Markdown).
    \usepackage{graphicx}
    % We will generate all images so they have a width \maxwidth. This means
    % that they will get their normal width if they fit onto the page, but
    % are scaled down if they would overflow the margins.
    \makeatletter
    \def\maxwidth{\ifdim\Gin@nat@width>\linewidth\linewidth
    \else\Gin@nat@width\fi}
    \makeatother
    \let\Oldincludegraphics\includegraphics
    % Set max figure width to be 80% of text width, for now hardcoded.
    \renewcommand{\includegraphics}[1]{\Oldincludegraphics[width=.8\maxwidth]{#1}}
    % Ensure that by default, figures have no caption (until we provide a
    % proper Figure object with a Caption API and a way to capture that
    % in the conversion process - todo).
    \usepackage{caption}
    \DeclareCaptionLabelFormat{nolabel}{}
    \captionsetup{labelformat=nolabel}

    \usepackage{adjustbox} % Used to constrain images to a maximum size 
    \usepackage{xcolor} % Allow colors to be defined
    \usepackage{enumerate} % Needed for markdown enumerations to work
    \usepackage{geometry} % Used to adjust the document margins
    \usepackage{amsmath} % Equations
    \usepackage{amssymb} % Equations
    \usepackage{textcomp} % defines textquotesingle
    % Hack from http://tex.stackexchange.com/a/47451/13684:
    \AtBeginDocument{%
        \def\PYZsq{\textquotesingle}% Upright quotes in Pygmentized code
    }
    \usepackage{upquote} % Upright quotes for verbatim code
    \usepackage{eurosym} % defines \euro
    \usepackage[mathletters]{ucs} % Extended unicode (utf-8) support
    \usepackage[utf8x]{inputenc} % Allow utf-8 characters in the tex document
    \usepackage{fancyvrb} % verbatim replacement that allows latex
    \usepackage{grffile} % extends the file name processing of package graphics 
                         % to support a larger range 
    % The hyperref package gives us a pdf with properly built
    % internal navigation ('pdf bookmarks' for the table of contents,
    % internal cross-reference links, web links for URLs, etc.)
    \usepackage{hyperref}
    \usepackage{longtable} % longtable support required by pandoc >1.10
    \usepackage{booktabs}  % table support for pandoc > 1.12.2
    \usepackage[inline]{enumitem} % IRkernel/repr support (it uses the enumerate* environment)
    \usepackage[normalem]{ulem} % ulem is needed to support strikethroughs (\sout)
                                % normalem makes italics be italics, not underlines
    

    
    
    % Colors for the hyperref package
    \definecolor{urlcolor}{rgb}{0,.145,.698}
    \definecolor{linkcolor}{rgb}{.71,0.21,0.01}
    \definecolor{citecolor}{rgb}{.12,.54,.11}

    % ANSI colors
    \definecolor{ansi-black}{HTML}{3E424D}
    \definecolor{ansi-black-intense}{HTML}{282C36}
    \definecolor{ansi-red}{HTML}{E75C58}
    \definecolor{ansi-red-intense}{HTML}{B22B31}
    \definecolor{ansi-green}{HTML}{00A250}
    \definecolor{ansi-green-intense}{HTML}{007427}
    \definecolor{ansi-yellow}{HTML}{DDB62B}
    \definecolor{ansi-yellow-intense}{HTML}{B27D12}
    \definecolor{ansi-blue}{HTML}{208FFB}
    \definecolor{ansi-blue-intense}{HTML}{0065CA}
    \definecolor{ansi-magenta}{HTML}{D160C4}
    \definecolor{ansi-magenta-intense}{HTML}{A03196}
    \definecolor{ansi-cyan}{HTML}{60C6C8}
    \definecolor{ansi-cyan-intense}{HTML}{258F8F}
    \definecolor{ansi-white}{HTML}{C5C1B4}
    \definecolor{ansi-white-intense}{HTML}{A1A6B2}

    % commands and environments needed by pandoc snippets
    % extracted from the output of `pandoc -s`
    \providecommand{\tightlist}{%
      \setlength{\itemsep}{0pt}\setlength{\parskip}{0pt}}
    \DefineVerbatimEnvironment{Highlighting}{Verbatim}{commandchars=\\\{\}}
    % Add ',fontsize=\small' for more characters per line
    \newenvironment{Shaded}{}{}
    \newcommand{\KeywordTok}[1]{\textcolor[rgb]{0.00,0.44,0.13}{\textbf{{#1}}}}
    \newcommand{\DataTypeTok}[1]{\textcolor[rgb]{0.56,0.13,0.00}{{#1}}}
    \newcommand{\DecValTok}[1]{\textcolor[rgb]{0.25,0.63,0.44}{{#1}}}
    \newcommand{\BaseNTok}[1]{\textcolor[rgb]{0.25,0.63,0.44}{{#1}}}
    \newcommand{\FloatTok}[1]{\textcolor[rgb]{0.25,0.63,0.44}{{#1}}}
    \newcommand{\CharTok}[1]{\textcolor[rgb]{0.25,0.44,0.63}{{#1}}}
    \newcommand{\StringTok}[1]{\textcolor[rgb]{0.25,0.44,0.63}{{#1}}}
    \newcommand{\CommentTok}[1]{\textcolor[rgb]{0.38,0.63,0.69}{\textit{{#1}}}}
    \newcommand{\OtherTok}[1]{\textcolor[rgb]{0.00,0.44,0.13}{{#1}}}
    \newcommand{\AlertTok}[1]{\textcolor[rgb]{1.00,0.00,0.00}{\textbf{{#1}}}}
    \newcommand{\FunctionTok}[1]{\textcolor[rgb]{0.02,0.16,0.49}{{#1}}}
    \newcommand{\RegionMarkerTok}[1]{{#1}}
    \newcommand{\ErrorTok}[1]{\textcolor[rgb]{1.00,0.00,0.00}{\textbf{{#1}}}}
    \newcommand{\NormalTok}[1]{{#1}}
    
    % Additional commands for more recent versions of Pandoc
    \newcommand{\ConstantTok}[1]{\textcolor[rgb]{0.53,0.00,0.00}{{#1}}}
    \newcommand{\SpecialCharTok}[1]{\textcolor[rgb]{0.25,0.44,0.63}{{#1}}}
    \newcommand{\VerbatimStringTok}[1]{\textcolor[rgb]{0.25,0.44,0.63}{{#1}}}
    \newcommand{\SpecialStringTok}[1]{\textcolor[rgb]{0.73,0.40,0.53}{{#1}}}
    \newcommand{\ImportTok}[1]{{#1}}
    \newcommand{\DocumentationTok}[1]{\textcolor[rgb]{0.73,0.13,0.13}{\textit{{#1}}}}
    \newcommand{\AnnotationTok}[1]{\textcolor[rgb]{0.38,0.63,0.69}{\textbf{\textit{{#1}}}}}
    \newcommand{\CommentVarTok}[1]{\textcolor[rgb]{0.38,0.63,0.69}{\textbf{\textit{{#1}}}}}
    \newcommand{\VariableTok}[1]{\textcolor[rgb]{0.10,0.09,0.49}{{#1}}}
    \newcommand{\ControlFlowTok}[1]{\textcolor[rgb]{0.00,0.44,0.13}{\textbf{{#1}}}}
    \newcommand{\OperatorTok}[1]{\textcolor[rgb]{0.40,0.40,0.40}{{#1}}}
    \newcommand{\BuiltInTok}[1]{{#1}}
    \newcommand{\ExtensionTok}[1]{{#1}}
    \newcommand{\PreprocessorTok}[1]{\textcolor[rgb]{0.74,0.48,0.00}{{#1}}}
    \newcommand{\AttributeTok}[1]{\textcolor[rgb]{0.49,0.56,0.16}{{#1}}}
    \newcommand{\InformationTok}[1]{\textcolor[rgb]{0.38,0.63,0.69}{\textbf{\textit{{#1}}}}}
    \newcommand{\WarningTok}[1]{\textcolor[rgb]{0.38,0.63,0.69}{\textbf{\textit{{#1}}}}}
    
    
    % Define a nice break command that doesn't care if a line doesn't already
    % exist.
    \def\br{\hspace*{\fill} \\* }
    % Math Jax compatability definitions
    \def\gt{>}
    \def\lt{<}
    % Document parameters
    \title{Writeup}
    
    
    

    % Pygments definitions
    
\makeatletter
\def\PY@reset{\let\PY@it=\relax \let\PY@bf=\relax%
    \let\PY@ul=\relax \let\PY@tc=\relax%
    \let\PY@bc=\relax \let\PY@ff=\relax}
\def\PY@tok#1{\csname PY@tok@#1\endcsname}
\def\PY@toks#1+{\ifx\relax#1\empty\else%
    \PY@tok{#1}\expandafter\PY@toks\fi}
\def\PY@do#1{\PY@bc{\PY@tc{\PY@ul{%
    \PY@it{\PY@bf{\PY@ff{#1}}}}}}}
\def\PY#1#2{\PY@reset\PY@toks#1+\relax+\PY@do{#2}}

\expandafter\def\csname PY@tok@sc\endcsname{\def\PY@tc##1{\textcolor[rgb]{0.73,0.13,0.13}{##1}}}
\expandafter\def\csname PY@tok@cm\endcsname{\let\PY@it=\textit\def\PY@tc##1{\textcolor[rgb]{0.25,0.50,0.50}{##1}}}
\expandafter\def\csname PY@tok@ow\endcsname{\let\PY@bf=\textbf\def\PY@tc##1{\textcolor[rgb]{0.67,0.13,1.00}{##1}}}
\expandafter\def\csname PY@tok@vm\endcsname{\def\PY@tc##1{\textcolor[rgb]{0.10,0.09,0.49}{##1}}}
\expandafter\def\csname PY@tok@nt\endcsname{\let\PY@bf=\textbf\def\PY@tc##1{\textcolor[rgb]{0.00,0.50,0.00}{##1}}}
\expandafter\def\csname PY@tok@cp\endcsname{\def\PY@tc##1{\textcolor[rgb]{0.74,0.48,0.00}{##1}}}
\expandafter\def\csname PY@tok@ch\endcsname{\let\PY@it=\textit\def\PY@tc##1{\textcolor[rgb]{0.25,0.50,0.50}{##1}}}
\expandafter\def\csname PY@tok@nd\endcsname{\def\PY@tc##1{\textcolor[rgb]{0.67,0.13,1.00}{##1}}}
\expandafter\def\csname PY@tok@dl\endcsname{\def\PY@tc##1{\textcolor[rgb]{0.73,0.13,0.13}{##1}}}
\expandafter\def\csname PY@tok@cpf\endcsname{\let\PY@it=\textit\def\PY@tc##1{\textcolor[rgb]{0.25,0.50,0.50}{##1}}}
\expandafter\def\csname PY@tok@si\endcsname{\let\PY@bf=\textbf\def\PY@tc##1{\textcolor[rgb]{0.73,0.40,0.53}{##1}}}
\expandafter\def\csname PY@tok@no\endcsname{\def\PY@tc##1{\textcolor[rgb]{0.53,0.00,0.00}{##1}}}
\expandafter\def\csname PY@tok@il\endcsname{\def\PY@tc##1{\textcolor[rgb]{0.40,0.40,0.40}{##1}}}
\expandafter\def\csname PY@tok@gi\endcsname{\def\PY@tc##1{\textcolor[rgb]{0.00,0.63,0.00}{##1}}}
\expandafter\def\csname PY@tok@gd\endcsname{\def\PY@tc##1{\textcolor[rgb]{0.63,0.00,0.00}{##1}}}
\expandafter\def\csname PY@tok@nc\endcsname{\let\PY@bf=\textbf\def\PY@tc##1{\textcolor[rgb]{0.00,0.00,1.00}{##1}}}
\expandafter\def\csname PY@tok@k\endcsname{\let\PY@bf=\textbf\def\PY@tc##1{\textcolor[rgb]{0.00,0.50,0.00}{##1}}}
\expandafter\def\csname PY@tok@se\endcsname{\let\PY@bf=\textbf\def\PY@tc##1{\textcolor[rgb]{0.73,0.40,0.13}{##1}}}
\expandafter\def\csname PY@tok@m\endcsname{\def\PY@tc##1{\textcolor[rgb]{0.40,0.40,0.40}{##1}}}
\expandafter\def\csname PY@tok@ge\endcsname{\let\PY@it=\textit}
\expandafter\def\csname PY@tok@gh\endcsname{\let\PY@bf=\textbf\def\PY@tc##1{\textcolor[rgb]{0.00,0.00,0.50}{##1}}}
\expandafter\def\csname PY@tok@bp\endcsname{\def\PY@tc##1{\textcolor[rgb]{0.00,0.50,0.00}{##1}}}
\expandafter\def\csname PY@tok@s\endcsname{\def\PY@tc##1{\textcolor[rgb]{0.73,0.13,0.13}{##1}}}
\expandafter\def\csname PY@tok@gp\endcsname{\let\PY@bf=\textbf\def\PY@tc##1{\textcolor[rgb]{0.00,0.00,0.50}{##1}}}
\expandafter\def\csname PY@tok@vg\endcsname{\def\PY@tc##1{\textcolor[rgb]{0.10,0.09,0.49}{##1}}}
\expandafter\def\csname PY@tok@mi\endcsname{\def\PY@tc##1{\textcolor[rgb]{0.40,0.40,0.40}{##1}}}
\expandafter\def\csname PY@tok@c\endcsname{\let\PY@it=\textit\def\PY@tc##1{\textcolor[rgb]{0.25,0.50,0.50}{##1}}}
\expandafter\def\csname PY@tok@nv\endcsname{\def\PY@tc##1{\textcolor[rgb]{0.10,0.09,0.49}{##1}}}
\expandafter\def\csname PY@tok@kt\endcsname{\def\PY@tc##1{\textcolor[rgb]{0.69,0.00,0.25}{##1}}}
\expandafter\def\csname PY@tok@ss\endcsname{\def\PY@tc##1{\textcolor[rgb]{0.10,0.09,0.49}{##1}}}
\expandafter\def\csname PY@tok@nn\endcsname{\let\PY@bf=\textbf\def\PY@tc##1{\textcolor[rgb]{0.00,0.00,1.00}{##1}}}
\expandafter\def\csname PY@tok@sd\endcsname{\let\PY@it=\textit\def\PY@tc##1{\textcolor[rgb]{0.73,0.13,0.13}{##1}}}
\expandafter\def\csname PY@tok@sh\endcsname{\def\PY@tc##1{\textcolor[rgb]{0.73,0.13,0.13}{##1}}}
\expandafter\def\csname PY@tok@nl\endcsname{\def\PY@tc##1{\textcolor[rgb]{0.63,0.63,0.00}{##1}}}
\expandafter\def\csname PY@tok@kp\endcsname{\def\PY@tc##1{\textcolor[rgb]{0.00,0.50,0.00}{##1}}}
\expandafter\def\csname PY@tok@kc\endcsname{\let\PY@bf=\textbf\def\PY@tc##1{\textcolor[rgb]{0.00,0.50,0.00}{##1}}}
\expandafter\def\csname PY@tok@c1\endcsname{\let\PY@it=\textit\def\PY@tc##1{\textcolor[rgb]{0.25,0.50,0.50}{##1}}}
\expandafter\def\csname PY@tok@gt\endcsname{\def\PY@tc##1{\textcolor[rgb]{0.00,0.27,0.87}{##1}}}
\expandafter\def\csname PY@tok@s1\endcsname{\def\PY@tc##1{\textcolor[rgb]{0.73,0.13,0.13}{##1}}}
\expandafter\def\csname PY@tok@ne\endcsname{\let\PY@bf=\textbf\def\PY@tc##1{\textcolor[rgb]{0.82,0.25,0.23}{##1}}}
\expandafter\def\csname PY@tok@cs\endcsname{\let\PY@it=\textit\def\PY@tc##1{\textcolor[rgb]{0.25,0.50,0.50}{##1}}}
\expandafter\def\csname PY@tok@s2\endcsname{\def\PY@tc##1{\textcolor[rgb]{0.73,0.13,0.13}{##1}}}
\expandafter\def\csname PY@tok@o\endcsname{\def\PY@tc##1{\textcolor[rgb]{0.40,0.40,0.40}{##1}}}
\expandafter\def\csname PY@tok@gu\endcsname{\let\PY@bf=\textbf\def\PY@tc##1{\textcolor[rgb]{0.50,0.00,0.50}{##1}}}
\expandafter\def\csname PY@tok@sb\endcsname{\def\PY@tc##1{\textcolor[rgb]{0.73,0.13,0.13}{##1}}}
\expandafter\def\csname PY@tok@gs\endcsname{\let\PY@bf=\textbf}
\expandafter\def\csname PY@tok@w\endcsname{\def\PY@tc##1{\textcolor[rgb]{0.73,0.73,0.73}{##1}}}
\expandafter\def\csname PY@tok@go\endcsname{\def\PY@tc##1{\textcolor[rgb]{0.53,0.53,0.53}{##1}}}
\expandafter\def\csname PY@tok@sr\endcsname{\def\PY@tc##1{\textcolor[rgb]{0.73,0.40,0.53}{##1}}}
\expandafter\def\csname PY@tok@kd\endcsname{\let\PY@bf=\textbf\def\PY@tc##1{\textcolor[rgb]{0.00,0.50,0.00}{##1}}}
\expandafter\def\csname PY@tok@mh\endcsname{\def\PY@tc##1{\textcolor[rgb]{0.40,0.40,0.40}{##1}}}
\expandafter\def\csname PY@tok@mf\endcsname{\def\PY@tc##1{\textcolor[rgb]{0.40,0.40,0.40}{##1}}}
\expandafter\def\csname PY@tok@ni\endcsname{\let\PY@bf=\textbf\def\PY@tc##1{\textcolor[rgb]{0.60,0.60,0.60}{##1}}}
\expandafter\def\csname PY@tok@vc\endcsname{\def\PY@tc##1{\textcolor[rgb]{0.10,0.09,0.49}{##1}}}
\expandafter\def\csname PY@tok@fm\endcsname{\def\PY@tc##1{\textcolor[rgb]{0.00,0.00,1.00}{##1}}}
\expandafter\def\csname PY@tok@nf\endcsname{\def\PY@tc##1{\textcolor[rgb]{0.00,0.00,1.00}{##1}}}
\expandafter\def\csname PY@tok@mo\endcsname{\def\PY@tc##1{\textcolor[rgb]{0.40,0.40,0.40}{##1}}}
\expandafter\def\csname PY@tok@na\endcsname{\def\PY@tc##1{\textcolor[rgb]{0.49,0.56,0.16}{##1}}}
\expandafter\def\csname PY@tok@gr\endcsname{\def\PY@tc##1{\textcolor[rgb]{1.00,0.00,0.00}{##1}}}
\expandafter\def\csname PY@tok@vi\endcsname{\def\PY@tc##1{\textcolor[rgb]{0.10,0.09,0.49}{##1}}}
\expandafter\def\csname PY@tok@err\endcsname{\def\PY@bc##1{\setlength{\fboxsep}{0pt}\fcolorbox[rgb]{1.00,0.00,0.00}{1,1,1}{\strut ##1}}}
\expandafter\def\csname PY@tok@mb\endcsname{\def\PY@tc##1{\textcolor[rgb]{0.40,0.40,0.40}{##1}}}
\expandafter\def\csname PY@tok@sa\endcsname{\def\PY@tc##1{\textcolor[rgb]{0.73,0.13,0.13}{##1}}}
\expandafter\def\csname PY@tok@kn\endcsname{\let\PY@bf=\textbf\def\PY@tc##1{\textcolor[rgb]{0.00,0.50,0.00}{##1}}}
\expandafter\def\csname PY@tok@sx\endcsname{\def\PY@tc##1{\textcolor[rgb]{0.00,0.50,0.00}{##1}}}
\expandafter\def\csname PY@tok@kr\endcsname{\let\PY@bf=\textbf\def\PY@tc##1{\textcolor[rgb]{0.00,0.50,0.00}{##1}}}
\expandafter\def\csname PY@tok@nb\endcsname{\def\PY@tc##1{\textcolor[rgb]{0.00,0.50,0.00}{##1}}}

\def\PYZbs{\char`\\}
\def\PYZus{\char`\_}
\def\PYZob{\char`\{}
\def\PYZcb{\char`\}}
\def\PYZca{\char`\^}
\def\PYZam{\char`\&}
\def\PYZlt{\char`\<}
\def\PYZgt{\char`\>}
\def\PYZsh{\char`\#}
\def\PYZpc{\char`\%}
\def\PYZdl{\char`\$}
\def\PYZhy{\char`\-}
\def\PYZsq{\char`\'}
\def\PYZdq{\char`\"}
\def\PYZti{\char`\~}
% for compatibility with earlier versions
\def\PYZat{@}
\def\PYZlb{[}
\def\PYZrb{]}
\makeatother


    % Exact colors from NB
    \definecolor{incolor}{rgb}{0.0, 0.0, 0.5}
    \definecolor{outcolor}{rgb}{0.545, 0.0, 0.0}



    
    % Prevent overflowing lines due to hard-to-break entities
    \sloppy 
    % Setup hyperref package
    \hypersetup{
      breaklinks=true,  % so long urls are correctly broken across lines
      colorlinks=true,
      urlcolor=urlcolor,
      linkcolor=linkcolor,
      citecolor=citecolor,
      }
    % Slightly bigger margins than the latex defaults
    
    \geometry{verbose,tmargin=1in,bmargin=1in,lmargin=1in,rmargin=1in}
    
    

    \begin{document}
    
    
    \maketitle
    
    

    
    \hypertarget{traffic-sign-recognition}{%
\section{\texorpdfstring{\textbf{Traffic Sign
Recognition}}{Traffic Sign Recognition}}\label{traffic-sign-recognition}}

\hypertarget{writeup}{%
\subsection{Writeup}\label{writeup}}

\begin{center}\rule{0.5\linewidth}{\linethickness}\end{center}

\textbf{Build a Traffic Sign Recognition Project}

The goals / steps of this project are the following: * Load the data set
(see below for links to the project data set) * Explore, summarize and
visualize the data set * Design, train and test a model architecture *
Use the model to make predictions on new images * Analyze the softmax
probabilities of the new images * Summarize the results with a written
report

\hypertarget{rubric-points}{%
\subsection{Rubric Points}\label{rubric-points}}

\hypertarget{here-i-will-consider-the-rubric-points-individually-and-describe-how-i-addressed-each-point-in-my-implementation.}{%
\subsection{\texorpdfstring{\#\#\# Here I will consider the
\href{https://review.udacity.com/\#!/rubrics/481/view}{rubric points}
individually and describe how I addressed each point in my
implementation.}{\#\#\# Here I will consider the rubric points individually and describe how I addressed each point in my implementation.}}\label{here-i-will-consider-the-rubric-points-individually-and-describe-how-i-addressed-each-point-in-my-implementation.}}

    \hypertarget{writeup-readme}{%
\subsubsection{Writeup / README}\label{writeup-readme}}

\hypertarget{provide-a-writeup-readme-that-includes-all-the-rubric-points-and-how-you-addressed-each-one.-you-can-submit-your-writeup-as-markdown-or-pdf.-you-can-use-this-template-as-a-guide-for-writing-the-report.-the-submission-includes-the-project-code.}{%
\paragraph{1. Provide a Writeup / README that includes all the rubric
points and how you addressed each one. You can submit your writeup as
markdown or pdf. You can use this template as a guide for writing the
report. The submission includes the project
code.}\label{provide-a-writeup-readme-that-includes-all-the-rubric-points-and-how-you-addressed-each-one.-you-can-submit-your-writeup-as-markdown-or-pdf.-you-can-use-this-template-as-a-guide-for-writing-the-report.-the-submission-includes-the-project-code.}}

You're reading it!

\hypertarget{data-set-summary-exploration}{%
\subsubsection{Data Set Summary \&
Exploration}\label{data-set-summary-exploration}}

\hypertarget{provide-a-basic-summary-of-the-data-set.-in-the-code-the-analysis-should-be-done-using-python-numpy-andor-pandas-methods-rather-than-hardcoding-results-manually.}{%
\paragraph{1. Provide a basic summary of the data set. In the code, the
analysis should be done using python, numpy and/or pandas methods rather
than hardcoding results
manually.}\label{provide-a-basic-summary-of-the-data-set.-in-the-code-the-analysis-should-be-done-using-python-numpy-andor-pandas-methods-rather-than-hardcoding-results-manually.}}

I used the pandas library to calculate summary statistics of the traffic
signs data set:

\begin{itemize}
\tightlist
\item
  The size of training set is 34799
\item
  The size of the validation set is 4410
\item
  The size of test set is 12630
\item
  The shape of a traffic sign image is (32, 32, 3)
\item
  The number of unique classes/labels in the data set is 43
\end{itemize}

\hypertarget{include-an-exploratory-visualization-of-the-dataset.}{%
\paragraph{2. Include an exploratory visualization of the
dataset.}\label{include-an-exploratory-visualization-of-the-dataset.}}

Here is an exploratory visualization of the data set. It is a bar chart
showing how the training data is distributed by class
\includegraphics{Nubmer of training examples per class.jpg}

    \hypertarget{design-and-test-a-model-architecture}{%
\subsubsection{Design and Test a Model
Architecture}\label{design-and-test-a-model-architecture}}

\hypertarget{describe-how-you-preprocessed-the-image-data.-what-techniques-were-chosen-and-why-did-you-choose-these-techniques-consider-including-images-showing-the-output-of-each-preprocessing-technique.-pre-processing-refers-to-techniques-such-as-converting-to-grayscale-normalization-etc.-optional-as-described-in-the-stand-out-suggestions-part-of-the-rubric-if-you-generated-additional-data-for-training-describe-why-you-decided-to-generate-additional-data-how-you-generated-the-data-and-provide-example-images-of-the-additional-data.-then-describe-the-characteristics-of-the-augmented-training-set-like-number-of-images-in-the-set-number-of-images-for-each-class-etc.}{%
\paragraph{1. Describe how you preprocessed the image data. What
techniques were chosen and why did you choose these techniques? Consider
including images showing the output of each preprocessing technique.
Pre-processing refers to techniques such as converting to grayscale,
normalization, etc. (OPTIONAL: As described in the ``Stand Out
Suggestions'' part of the rubric, if you generated additional data for
training, describe why you decided to generate additional data, how you
generated the data, and provide example images of the additional data.
Then describe the characteristics of the augmented training set like
number of images in the set, number of images for each class,
etc.)}\label{describe-how-you-preprocessed-the-image-data.-what-techniques-were-chosen-and-why-did-you-choose-these-techniques-consider-including-images-showing-the-output-of-each-preprocessing-technique.-pre-processing-refers-to-techniques-such-as-converting-to-grayscale-normalization-etc.-optional-as-described-in-the-stand-out-suggestions-part-of-the-rubric-if-you-generated-additional-data-for-training-describe-why-you-decided-to-generate-additional-data-how-you-generated-the-data-and-provide-example-images-of-the-additional-data.-then-describe-the-characteristics-of-the-augmented-training-set-like-number-of-images-in-the-set-number-of-images-for-each-class-etc.}}

As a first step, I decided to convert the images to grayscale because
lighting and color adds too much variation to examples, and it will
require much more examples to train the model

Here is an example of a traffic sign image before and after grayscaling.

\includegraphics{imageOriginal.png} \includegraphics{imageGrayscale.png}

As a last step, I normalized the image data because because this way it
is easier to train the model

    \hypertarget{describe-what-your-final-model-architecture-looks-like-including-model-type-layers-layer-sizes-connectivity-etc.-consider-including-a-diagram-andor-table-describing-the-final-model.}{%
\paragraph{2. Describe what your final model architecture looks like
including model type, layers, layer sizes, connectivity, etc.) Consider
including a diagram and/or table describing the final
model.}\label{describe-what-your-final-model-architecture-looks-like-including-model-type-layers-layer-sizes-connectivity-etc.-consider-including-a-diagram-andor-table-describing-the-final-model.}}

My final model consisted of the following layers:

\begin{longtable}[]{@{}cc@{}}
\toprule
Layer & Description\tabularnewline
\midrule
\endhead
Input & 32x32x1 grayscale image\tabularnewline
Convolution 3x3(1) & 1x1 stride, valid padding, outputs
28x28x128\tabularnewline
RELU &\tabularnewline
Dropout & keep probability .5\tabularnewline
Max pooling(1) & 2x2 stride, outputs 14x14x128\tabularnewline
Convolution 3x3(2) & 1x1 stride, valid padding, outputs
10x10x128\tabularnewline
Max pooling(2) & 2x2 stride, outputs 5x5x128\tabularnewline
Max pooling(3) & takes Max pooling(1) as input, 2x2
stride,\tabularnewline
& outputs 7x7x128\tabularnewline
Fully connected(1) & Takes Max pooling(2) and Max pooling(3)
as\tabularnewline
& input, outputs 100\tabularnewline
Fully connected(2) &\tabularnewline
Softmax &\tabularnewline
\bottomrule
\end{longtable}

    \hypertarget{describe-how-you-trained-your-model.-the-discussion-can-include-the-type-of-optimizer-the-batch-size-number-of-epochs-and-any-hyperparameters-such-as-learning-rate.}{%
\paragraph{3. Describe how you trained your model. The discussion can
include the type of optimizer, the batch size, number of epochs and any
hyperparameters such as learning
rate.}\label{describe-how-you-trained-your-model.-the-discussion-can-include-the-type-of-optimizer-the-batch-size-number-of-epochs-and-any-hyperparameters-such-as-learning-rate.}}

To train the model, I used .5 as keep probabily parameter for dropout.
The Adam algorithm is used for optimization. The model is trained for
100 epochs, with learning rate 0.001 and batch size 128.

    \hypertarget{describe-the-approach-taken-for-finding-a-solution-and-getting-the-validation-set-accuracy-to-be-at-least-0.93.-include-in-the-discussion-the-results-on-the-training-validation-and-test-sets-and-where-in-the-code-these-were-calculated.-your-approach-may-have-been-an-iterative-process-in-which-case-outline-the-steps-you-took-to-get-to-the-final-solution-and-why-you-chose-those-steps.-perhaps-your-solution-involved-an-already-well-known-implementation-or-architecture.-in-this-case-discuss-why-you-think-the-architecture-is-suitable-for-the-current-problem.}{%
\paragraph{4. Describe the approach taken for finding a solution and
getting the validation set accuracy to be at least 0.93. Include in the
discussion the results on the training, validation and test sets and
where in the code these were calculated. Your approach may have been an
iterative process, in which case, outline the steps you took to get to
the final solution and why you chose those steps. Perhaps your solution
involved an already well known implementation or architecture. In this
case, discuss why you think the architecture is suitable for the current
problem.}\label{describe-the-approach-taken-for-finding-a-solution-and-getting-the-validation-set-accuracy-to-be-at-least-0.93.-include-in-the-discussion-the-results-on-the-training-validation-and-test-sets-and-where-in-the-code-these-were-calculated.-your-approach-may-have-been-an-iterative-process-in-which-case-outline-the-steps-you-took-to-get-to-the-final-solution-and-why-you-chose-those-steps.-perhaps-your-solution-involved-an-already-well-known-implementation-or-architecture.-in-this-case-discuss-why-you-think-the-architecture-is-suitable-for-the-current-problem.}}

My final model results were: * training set accuracy of 100\% *
validation set accuracy of 96.3\% * test set accuracy of 94.4\%

If a well known architecture was chosen: * What architecture was chosen?
I used architecture described in this paper
http://yann.lecun.com/exdb/publis/pdf/sermanet-ijcnn-11.pdf . It is a
traditional ConvNet architecture, modified by feeding 1st stage features
in addition to 2nd stage features to the classifier. * Why did you
believe it would be relevant to the traffic sign application? This
architecture was applied to the task of traffic sign classification as
part of the GTSRB competition. So I decided to just copy it for my
project. * How does the final model's accuracy on the training,
validation and test set provide evidence that the model is working well?
Original architecture has reached 98.97\% accuracy on traffic sign
recognition task.

    \hypertarget{test-a-model-on-new-images}{%
\subsubsection{Test a Model on New
Images}\label{test-a-model-on-new-images}}

\hypertarget{choose-five-german-traffic-signs-found-on-the-web-and-provide-them-in-the-report.-for-each-image-discuss-what-quality-or-qualities-might-be-difficult-to-classify.}{%
\paragraph{1. Choose five German traffic signs found on the web and
provide them in the report. For each image, discuss what quality or
qualities might be difficult to
classify.}\label{choose-five-german-traffic-signs-found-on-the-web-and-provide-them-in-the-report.-for-each-image-discuss-what-quality-or-qualities-might-be-difficult-to-classify.}}

Here are five German traffic signs that I found on the web:

\includegraphics{sign examples/image1.jpg}
\includegraphics{sign examples/image2.jpg}
\includegraphics{sign examples/image3.jpg}
\includegraphics{sign examples/image4.jpg}
\includegraphics{sign examples/image5.jpg}

``Speed limit (70km/h)'' sign might be difficult to classify because is
similar to ``Speed limit (20km/h)'' sign ``Children crossing'' sign
might be difficult to clasify because of complex people's figures
``Ahead only'' is very similar to other arrow-like signs ``Priority
road'' has very little features, except romb shape and color ``Stop''
sing has letters, which may be difficult to recognize at low resolution

    \hypertarget{discuss-the-models-predictions-on-these-new-traffic-signs-and-compare-the-results-to-predicting-on-the-test-set.-at-a-minimum-discuss-what-the-predictions-were-the-accuracy-on-these-new-predictions-and-compare-the-accuracy-to-the-accuracy-on-the-test-set-optional-discuss-the-results-in-more-detail-as-described-in-the-stand-out-suggestions-part-of-the-rubric.}{%
\paragraph{2. Discuss the model's predictions on these new traffic signs
and compare the results to predicting on the test set. At a minimum,
discuss what the predictions were, the accuracy on these new
predictions, and compare the accuracy to the accuracy on the test set
(OPTIONAL: Discuss the results in more detail as described in the
``Stand Out Suggestions'' part of the
rubric).}\label{discuss-the-models-predictions-on-these-new-traffic-signs-and-compare-the-results-to-predicting-on-the-test-set.-at-a-minimum-discuss-what-the-predictions-were-the-accuracy-on-these-new-predictions-and-compare-the-accuracy-to-the-accuracy-on-the-test-set-optional-discuss-the-results-in-more-detail-as-described-in-the-stand-out-suggestions-part-of-the-rubric.}}

Here are the results of the prediction:

\begin{longtable}[]{@{}cc@{}}
\toprule
Image & Prediction\tabularnewline
\midrule
\endhead
Speed limit (70km/h) & Speed limit (20km/h)\tabularnewline
Children crossing & Right-of-way at the next intersection\tabularnewline
Ahead only & Ahead only\tabularnewline
Priority road & Priority road\tabularnewline
Stop & Speed limit (60km/h)\tabularnewline
\bottomrule
\end{longtable}

The model was able to correctly guess 2 of the 5 traffic signs, which
gives an accuracy of 40\%. This is much worse than 94.4\% that I got on
test set.

    \hypertarget{describe-how-certain-the-model-is-when-predicting-on-each-of-the-five-new-images-by-looking-at-the-softmax-probabilities-for-each-prediction.-provide-the-top-5-softmax-probabilities-for-each-image-along-with-the-sign-type-of-each-probability.-optional-as-described-in-the-stand-out-suggestions-part-of-the-rubric-visualizations-can-also-be-provided-such-as-bar-charts}{%
\paragraph{3. Describe how certain the model is when predicting on each
of the five new images by looking at the softmax probabilities for each
prediction. Provide the top 5 softmax probabilities for each image along
with the sign type of each probability. (OPTIONAL: as described in the
``Stand Out Suggestions'' part of the rubric, visualizations can also be
provided such as bar
charts)}\label{describe-how-certain-the-model-is-when-predicting-on-each-of-the-five-new-images-by-looking-at-the-softmax-probabilities-for-each-prediction.-provide-the-top-5-softmax-probabilities-for-each-image-along-with-the-sign-type-of-each-probability.-optional-as-described-in-the-stand-out-suggestions-part-of-the-rubric-visualizations-can-also-be-provided-such-as-bar-charts}}

The code for making predictions on my final model is located in the 51th
cell of the Ipython notebook.

For the first image, the model is completely sure that this is a Speed
limit (20km/h) sign (probability of .99), but image shows Speed limit
(70km/h) sign. The top five soft max probabilities were

\begin{longtable}[]{@{}cc@{}}
\toprule
Probability & Prediction\tabularnewline
\midrule
\endhead
.99 & Speed limit (20km/h)\tabularnewline
5.87080023e-04 & Speed limit (70km/h)\tabularnewline
9.91532989e-09 & Speed limit (30km/h)\tabularnewline
2.63868823e-11 & Speed limit (60km/h)\tabularnewline
2.00707669e-13 & Speed limit (120km/h)\tabularnewline
\bottomrule
\end{longtable}

For the second image, the model is completely sure that this is a
Right-of-way at the next intersection sign (probability of 0.99), but
the image shows a Stop sign. The top five soft max probabilities were

\begin{longtable}[]{@{}cc@{}}
\toprule
Probability & Prediction\tabularnewline
\midrule
\endhead
.99 & Right-of-way at the next intersection\tabularnewline
6.50445713e-07 & Children crossing\tabularnewline
6.25950625e-09 & Beware of ice/snow\tabularnewline
3.72544395e-09 & Dangerous curve to the left\tabularnewline
1.05222726e-10 & Slippery road\tabularnewline
\bottomrule
\end{longtable}

For the third image, the model is completely sure that this is a Ahead
only sign (probability of 0.99), and the prediction is correct. The top
five soft max probabilities were

\begin{longtable}[]{@{}cc@{}}
\toprule
Probability & Prediction\tabularnewline
\midrule
\endhead
0.99 & Ahead only\tabularnewline
1.3928113e-10 & Yield\tabularnewline
1.1458318e-12 & Priority road\tabularnewline
1.3326577e-13 & Speed limit (60km/h)\tabularnewline
1.7614978e-14 & No passing\tabularnewline
\bottomrule
\end{longtable}

For the forth image, the model is completely sure that this is a
Priority road sign (probability of 0.99), and the prediction is correct.
The top five soft max probabilities were

\begin{longtable}[]{@{}cc@{}}
\toprule
Probability & Prediction\tabularnewline
\midrule
\endhead
0.99 & Priority road\tabularnewline
4.1416447e-15 & Ahead only\tabularnewline
6.37214e-17 & Stop\tabularnewline
3.375843e-21 & No passing\tabularnewline
9.747884e-22 & Children crossing\tabularnewline
\bottomrule
\end{longtable}

For the fifth image, the model is almost sure that this is a Speed limit
(60km/h) sign (probability of 0.95), but the image shows a Stop sign.
The top five soft max probabilities were

\begin{longtable}[]{@{}cc@{}}
\toprule
Probability & Prediction\tabularnewline
\midrule
\endhead
0.9501321 & Speed limit (60km/h)\tabularnewline
0.048662927 & Stop\tabularnewline
0.0011762282 & Dangerous curve to the left\tabularnewline
1.18491635e-05 & No entry\tabularnewline
9.463262e-06 & Go straight or left\tabularnewline
\bottomrule
\end{longtable}


    % Add a bibliography block to the postdoc
    
    
    
    \end{document}
